\chapter{Introduction}

Nuclear reactors remain a highly debated topic in both scientific and political circles. While they play a crucial role in generating clean energy, significant challenges related to safety, environmental impact, and economics persist. For instance, the time and capital required to deploy a nuclear reactor are substantial \cite{Kornecki_Wise_2024}. This thesis aims to examine the current state of nuclear reactors, focusing on their design, operations, and the potential of thorium fuel cycles. It will explore the benefits and drawbacks of various nuclear fuel cycles and reactor designs, as well as their future prospects.

Nuclear energy is often associated with atomic weapons due to the events of World War II and the subsequent arms race during the Cold War \cite{Lamarsh_Baratta_2009}. However, following the end of the Cold War, the focus of the nuclear industry shifted towards civilian applications, particularly energy production, largely driven by growing concerns over greenhouse gas emissions. One of the key advantages of nuclear energy is that it produces power without emitting greenhouse gases \cite{Lamarsh_Baratta_2009}.

The development of nuclear reactors began in the mid-20th century, initially driven by military needs, particularly for the propulsion of naval vessels, such as submarines and aircraft carriers. Nuclear power offered significant advantages, especially in submarines, where the lack of a need for combustion allowed them to operate underwater for extended periods. In the case of aircraft carriers, nuclear propulsion freed up space previously used for fuel oil, allowing more room for aviation fuel and other supplies \cite{Lamarsh_Baratta_2009}.

Several countries explored the use of nuclear reactors for civilian ship propulsion. The U.S. nuclear-powered merchant ship, `Savannah', operated briefly during the 1960s and 1970s, demonstrating the technical feasibility of nuclear propulsion for commercial vessels, though it proved economically unviable. Other nations, including Japan, Germany, and the Soviet Union, also experimented with nuclear-powered merchant ships. Notably, Germany's ore carrier, `Otto Hahn', operated successfully for a decade before being decommissioned due to high costs. In contrast, the Soviet Union's icebreaker `Lenin' demonstrated the value of nuclear power in extreme environments, leading to the construction of additional nuclear-powered icebreakers \cite{Lamarsh_Baratta_2009}.

Nuclear power was also explored for other applications. In 1945, Eugene Wigner and Alvin Weinberg proposed a liquid fuel reactor design \cite{TMSR_book}. The Oak Ridge National Laboratory (ORNL) investigated a potential nuclear application of this concept to jet engines \cite{TMSR_book}. From 1949 to 1961, the U.S. invested approximately one billion dollars in the Aircraft Nuclear Propulsion Project (ANP). This project aimed to develop a nuclear-powered aircraft capable of long-range missions, a crucial priority during the early Cold War. However, with the advent of long-range ballistic missiles, the need for such an aircraft diminished, and the project was ultimately terminated \cite{Lamarsh_Baratta_2009}.

At the time, there were concerns that the global supply of \(\prescript{235}{}{U}\) would be insufficient to meet future energy demands. Weinberg and his colleagues recognized that molten salt reactors (MSRs) could be used to breed \(\prescript{233}{}{U}\) from thorium \cite{TMSR_book}. Between 1965 and 1969, ORNL successfully operated an MSR experiment for 15 months, achieving an impressive \(87\%\) operational uptime, even utilizing \(\prescript{233}{}{U}\) as fuel \cite{TMSR_book}. Despite this success, the MSR program was eventually discontinued as other nuclear concepts received greater political support \cite{TMSR_book}.

This thesis aims to explore the potential of the thorium fuel cycle in the nuclear industry and its prospects for the near future. The research investigates the benefits and challenges associated with thorium fuel, as well as the various proposed implementations for integrating this cycle into commercial reactors.


