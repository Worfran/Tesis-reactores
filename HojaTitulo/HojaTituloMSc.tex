%\newpage
%\setcounter{page}{1}

\begin{center}
\begin{figure*}
    \centering%
    \includegraphics[scale=0.35]{HojaTitulo/Figures_HojaTitulo/Universidad_de_los_Andes_(logo).png} % Replace \epsfig with \includegraphics and remove the ".eps" extension
\end{figure*}
\thispagestyle{empty} \vspace*{2.0cm} \textbf{\LARGE
The Thorium fuel cycle in nuclear reactors}\\[2.5cm]

\Large\textbf{Frank Worman Garcia Eslava}\\[2.0cm]

Advisor: PhD. Juan Carlos Sanabria Arenas \\[2.0cm]


\Large Thesis presented to obtain the degree of Physicist \\ [2.0cm]


Bogot\'{a}, Colombia\\ [0.5cm]
\today \\
\end{center}

\newpage{\pagestyle{empty}\cleardoublepage}

\newpage
\thispagestyle{empty} \textbf{}\normalsize
\\\\\\%
\textbf{\LARGE Regards}\\

\addcontentsline{toc}{chapter}{Regards}


\newpage

\textbf{\LARGE Resumen}
\addcontentsline{toc}{chapter}{Abstract}\\
Esta tesis explora el potencial de los combustibles nucleares basados en torio en los Reactores de Agua Presurizada (PWRs), enfocándose en el ciclo del combustible de torio y sus diversas implementaciones. La investigación investiga los beneficios y desafíos asociados con el combustible de torio, incluyendo sus mayores ratios de conversión, mejores propiedades térmicas y resistencia intrínseca a la proliferación. El estudio incluye simulaciones detalladas utilizando OpenMC para analizar el comportamiento del óxido de torio (ThOX) con uranio-233 (\(\prescript{233}{}{U}\)) en diferentes concentraciones. Los resultados demuestran que, aunque los combustibles basados en torio pueden lograr la reproducción, mantener la criticidad presenta desafíos. La tesis también examina el impacto de la composición y concentración del combustible en el rendimiento y la seguridad del reactor. Los hallazgos sugieren que optimizar estos parámetros es crucial para mejorar el rendimiento y la seguridad de los combustibles basados en torio. La investigación concluye con una discusión sobre las perspectivas futuras del torio en la industria nuclear y la necesidad de más investigación y desarrollo para realizar plenamente su potencial como un combustible nuclear sostenible.

\vspace{1.0cm}

\textbf{\small Palabras clave:} Ciclo del combustible de torio, Reactor de Agua Presurizada, Uranio-233, Combustible nuclear, Simulación de reactores, OpenMC, Energía sostenible, Reactores de Sales Fundidas de Torio\\

\newpage
\textbf{\LARGE Abstract}\\

This thesis explores the potential of thorium-based nuclear fuels in Pressurized Water Reactors (PWRs), focusing on the thorium fuel cycle and its various implementations. The research investigates the benefits and challenges associated with thorium fuel, including its higher conversion ratios, improved thermal properties, and intrinsic proliferation resistance. The study includes detailed simulations using OpenMC to analyze the behavior of thorium oxide (ThOX) with uranium-233 (\(\prescript{233}{}{U}\)) at different concentrations. The results demonstrate that while thorium-based fuels can achieve breeding, maintaining criticality presents challenges. The thesis also examines the impact of fuel composition and concentration on reactor performance and safety. The findings suggest that optimizing these parameters is crucial for enhancing the performance and safety of thorium-based fuels. The research concludes with a discussion on the future prospects of thorium in the nuclear industry and the need for further research and development to fully realize its potential as a sustainable nuclear fuel.

\vspace{1.0cm}

\textbf{\small Keywords:} Thorium fuel cycle, Pressurized Water Reactor, Uranium-233, Nuclear fuel, Reactor simulation, OpenMC, Sustainable energy, Thorium Molten Salt Reactors\\