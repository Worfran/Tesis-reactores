%\newpage
%\setcounter{page}{1}

\begin{center}
\begin{figure*}
    \centering%
    \includegraphics[scale=0.35]{HojaTitulo/Figures_HojaTitulo/Universidad_de_los_Andes_(logo).png} % Replace \epsfig with \includegraphics and remove the ".eps" extension
\end{figure*}
\thispagestyle{empty} \vspace*{2.0cm} \textbf{\LARGE
The Thorium fuel cycle in nuclear reactors}\\[2.5cm]

\Large\textbf{Frank Worman Garcia Eslava}\\[2.0cm]

Advisor: PhD. Juan Carlos Sanabria Arenas \\[2.0cm]


\Large Thesis presented to obtain the degree of Physicist \\ [2.0cm]


Bogot\'{a}, Colombia\\ [0.5cm]
\today \\
\end{center}

\newpage{\pagestyle{empty}\cleardoublepage}

\newpage
\thispagestyle{empty} \textbf{}\normalsize
\\\\\\%
\textbf{\LARGE Agradecimientos}\\
\addcontentsline{toc}{chapter}{Agradecimientos}

Quisiera comenzar agradeciendo a Dios y a mi familia por su amor, apoyo y paciencia durante todo este tiempo. A mi madre, por ser mi fuente de inspiración y por brindarme siempre su amor incondicional. A mi padre, por su sabiduría y su ejemplo de trabajo duro y dedicación, enseñándome el valor del esfuerzo constante. Ambos padres me han dado amor incondicional y son mi fuente de inspiración.

A mi hermana, primos y tíos por su apoyo, escucha y amor. En especial, a mis primos Jefferson y Yeiri, y a mi tía Jesusa, por estar siempre presentes y ser un pilar fundamental en mi vida.

A mis profesores, quienes han estado apoyándome a lo largo de toda mi carrera, especialmente a Juan Carlos Sanabria por su paciencia, dedicación y apoyo, siendo una figura supremamente importante a la hora de inspirarme con sus clases. A la profesora Yenny Rocío Hernández, quien me permitió realizar este proyecto y me brindó su guía invaluable.

A mi pareja, Luisa, por su amor incondicional, su apoyo constante y su infinita paciencia, siendo mi compañera fiel en cada paso de este camino. A mis amigos, por estar siempre a mi lado, compartiendo alegrías y enfrentando desafíos juntos. A Santiago Henano, por su invaluable ayuda en mis momentos más complicados con la programación. A Gabriel Villabon, por motivarme y recordarme siempre la importancia de dar lo mejor de mí. A Rafael Velásquez, por ser ese amigo que siempre está dispuesto a escuchar mis ideas, sin importar cuán descabelladas parezcan. A todos mis amigos de Prisma: Alejandra, Sara, Fabio, Juan Andres, Juan Pablo, y a tantos otros que no puedo mencionar individualmente, pero que son igual de importantes, gracias por su amistad y por permitirme vivir momentos inolvidables llenos de risas y buenos recuerdos.

Gracias a todos ustedes por su amor, apoyo y paciencia. Este logro no habría sido posible sin ustedes.
\newpage

\textbf{\LARGE Resumen}
\addcontentsline{toc}{chapter}{Abstract}\\
Esta tesis explora el potencial de los combustibles nucleares basados en torio en los Reactores de Agua Presurizada (PWRs), enfocándose en el ciclo del combustible de torio y sus diversas implementaciones. El proyecto investiga los beneficios y desafíos asociados con el combustible de torio, incluyendo sus mayores ratios de conversión, mejores propiedades térmicas y resistencia intrínseca a la proliferación. El estudio incluye simulaciones detalladas utilizando OpenMC para analizar el comportamiento del óxido de torio (ThOX) con uranio-233 (\(\prescript{233}{}{U}\)) en diferentes concentraciones. Los resultados demuestran que, aunque los combustibles basados en torio pueden lograr tasas de producción de isotopos fisiles positivas, mantener la criticidad presenta desafíos. La tesis también examina el impacto de la composición y concentración del combustible en el rendimiento y la seguridad del reactor. Los hallazgos sugieren que optimizar estos parámetros es crucial para mejorar el rendimiento y la seguridad de los combustibles basados en torio. La investigación concluye con una discusión sobre las perspectivas futuras del torio en la industria nuclear y la necesidad de más investigación y desarrollo para realizar plenamente su potencial como un combustible nuclear sostenible.

\vspace{1.0cm}

\textbf{\small Palabras clave:} Ciclo del combustible de torio, Reactor de Agua Presurizada, Uranio-233, Combustible nuclear, Simulación de reactores, OpenMC, Energía sostenible, Reactores de Sales Fundidas de Torio.\\

\newpage
\textbf{\LARGE Abstract}\\

This thesis explores the potential of thorium-based nuclear fuels in Pressurized Water Reactors (PWRs), focusing on the thorium fuel cycle and its various implementations. The research investigates the benefits and challenges associated with thorium fuel, including its higher conversion ratios, improved thermal properties, and intrinsic proliferation resistance. The study includes detailed simulations using OpenMC to analyze the behavior of thorium oxide (ThOX) with uranium-233 (\(\prescript{233}{}{U}\)) at different concentrations. The results demonstrate that while thorium-based fuels can achieve breeding, maintaining criticality presents challenges. The thesis also examines the impact of fuel composition and concentration on reactor performance and safety. The findings suggest that optimizing these parameters is crucial for enhancing the performance and safety of thorium-based fuels. The research concludes with a discussion on the future prospects of thorium in the nuclear industry and the need for further research and development to fully realize its potential as a sustainable nuclear fuel.

\vspace{1.0cm}

\textbf{\small Keywords:} Thorium fuel cycle, Pressurized Water Reactor, Uranium-233, Nuclear fuel, Reactor simulation, OpenMC, Sustainable energy, Thorium Molten Salt Reactors.\\