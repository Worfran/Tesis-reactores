\chapter{Present and future of Thorium}

The research and simulations conducted in this project provide valuable insights into the potential and challenges of using thorium-based nuclear fuels in PWRs. The thorium fuel cycle, with its unique properties and benefits, presents a promising alternative to traditional uranium-based fuels. However, several technical and operational challenges must be addressed to fully realize its potential.

\section{Thorium Fuel Cycle}

The thorium fuel cycle offers several advantages over the conventional uranium fuel cycle. Thorium is more abundant than uranium, and its use in nuclear reactors can lead to higher efficiency in harvesting energy from the fuel. The chemical stability and higher thermal conductivity of thorium dioxide (\(ThO_2\)) compared to uranium dioxide (\(UO_2\)) make it a more suitable material for nuclear fuel, potentially improving in-pile performance and long-term storage of irradiated fuel.

The intrinsic proliferation resistance of the thorium fuel cycle, due to the formation of \(\prescript{232}{}{U}\) and its strong gamma emissions, makes it less attractive for the production of nuclear weapons. This adds a significant layer of security in the context of nuclear non-proliferation.

\section{Simulation Results}

The simulations conducted using OpenMC provided detailed insights into the behavior of thorium-based fuels in a PWR. The results demonstrated that thorium oxide (ThOX) with uranium-233 (\(\prescript{233}{}{U}\)) at concentrations of \(5 \, \%\) and \(10 \, \%\) cannot achieve breeding. Thi is consistent with previous studies that have shown that breeding with ThOX and \(\prescript{233}{}{U}\) is challenging in thermal reactors \cite{roadmap}. However, it may be possible to achieve breeding with more sophisticated simulations and optimization of fuel composition and core configuration.

Simulations also show that with \(ThO_2\) in combination with \(\prescript{239}{}{Pu}\) a breeding ration can be achieved. However, the saturation poitn of the breeding process is achieved much later compared to \(\prescript{239}{}{Pu}\) breeding using uranium oxide. This delay is due to the slower breeding process of \(\prescript{233}{}{U}\) from \(\prescript{232}{}{Th}\) compared to the breeding of \(\prescript{239}{}{Pu}\) from \(\prescript{238}{}{U}\).

The simulations also showed an improvement in neutronic economy with thorium-based fuels. This improvement is characterized by a reduction in excess reactivity at the beginning of the fuel cycle, which helps to minimize the initial reactivity spike and associated control challenges. Furthermore, the simulations indicate that thorium-based fuels can maintain criticality towards the end of the fuel cycle, ensuring a more stable and efficient reactor operation over time. This behavior is consistent with the finding in \textbf{Ref.}\cite{LAU201248}

\section{Challenges and Future Research}

Despite the promising results, several challenges remain in the implementation of thorium-based fuels. Achieving and maintaining criticality with ThOX and \(\prescript{233}{}{U}\) requires careful consideration of fuel composition and concentration. The simulations indicated that higher thorium concentrations could lead to subcritical reactor behavior.

Additionally, achieving optimal breeding ratios with ThOX and \(\prescript{233}{}{U}\) may require more sophisticated designs in the fuel lattice and core configuration. The results suggest that optimizing these parameters is crucial for enhancing the performance and safety of thorium-based fuels.

Further research and development are necessary to address these challenges and fully realize the potential of thorium as a sustainable nuclear fuel. This includes exploring advanced reactor designs, such as Molten Salt Reactors (MSRs), which offer several advantages over traditional Light Water Reactors (LWRs), including higher thermal efficiency, lower proliferation risk, and improved safety features.

Additionally, it is essential to develop fuel reprocessing and recycling technologies to economically extract and reuse valuable fissile materials from irradiated thorium-based fuels. This will help to reduce the volume of radioactive waste and maximize the energy potential of thorium resources.

Finally, regulatory and policy frameworks must be established to support the deployment of thorium-based fuels and ensure their safe and secure operation. This includes addressing licensing and permitting requirements, as well as establishing international cooperation and standards for the use of thorium in nuclear energy production.

\section{Nuclear energy in Colombia}

The uranium and thorium reserves in Colombia offer a promising opportunity for energy generation and integration into the global nuclear fuel supply chain. These reserves could be harnessed to support the country's energy needs while contributing to the international market. However, any exploitation of these nuclear resources must prioritize sustainability, minimizing environmental impacts \cite{UraniumGlobal,ReservasUranio}.

To support the development of nuclear energy, it is crucial to promote education and public awareness. Additionally, continuous training and qualification of personnel operating and maintaining nuclear plants are vital to ensure operational safety.

Effective waste management strategies are imperative for the long-term sustainability of nuclear energy in Colombia. Establishing a comprehensive approach to managing radioactive waste, including researching solutions such as deep geological repositories, is necessary to address this challenge.

International collaboration can provide Colombia with access to valuable experience, resources, and best practices in the nuclear field. Exploring economic opportunities, such as the potential export of nuclear technology and knowledge, can further enhance the country's position in the global market.

While considering nuclear energy, Colombia should continue investing in renewable energies to reduce dependence on water resources and diversify its energy matrix. Ongoing research into the environmental impact of energy sources and exploring how nuclear energy can coexist sustainably with other sources is essential.

Strengthening Colombia's energy infrastructure and regulatory framework is crucial to ensure the safety, reliability, and sustainability of the nuclear sector. Citizen participation in decision-making processes are also important to build public trust and support for nuclear energy initiatives.

Nuclear energy offers a valuable opportunity for Colombia's energy future. By addressing safety, public acceptance, and waste management challenges, and by leveraging its uranium and thorium reserves, Colombia can make significant strides in its nuclear energy development. The thorium fuel cycle, in particular, presents several advantages, such as higher thermal conductivity, lower thermal expansion, and better in-pile performance compared to uranium dioxide.

\section{Conclusion}

In conclusion, thorium-based fuels, particularly ThOX with \(\prescript{233}{}{U}\), show significant promise for use in future reactors. The simulations conducted in this project provide valuable insights into the behavior of these fuels and highlight the potential benefits and challenges associated with their use. While the thorium fuel cycle offers several advantages, including higher conversion ratios, improved thermal properties, and intrinsic proliferation resistance, achieving and maintaining criticality presents challenges that require further research and optimization.

The future of thorium-based nuclear fuels looks promising, with ongoing research and development efforts aimed at overcoming the technical and operational challenges. By addressing these challenges, thorium-based fuels could play a crucial role in the future of sustainable and safe nuclear energy production.