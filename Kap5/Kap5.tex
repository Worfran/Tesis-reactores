\chapter{Nuclear fuel cycles}

Now we will examine the life cycle of the materials used as fuel in nuclear reactors. The fuel cycle spans from the mining of the ore to the fabrication of the fuel, its irradiation in the reactor, and the subsequent processing of the spent nuclear fuel \cite{fuel_cycle_book}. The nuclear fuel cycle can vary depending on several factors, such as:

\begin{itemize}
    \item The fissile nuclei being used for energy generation, such as any of the isotopes mentioned in \textbf{Section} \ref{sec:possible_fuels}.
    \item The form in which the fuel is utilized.
    \item The type of reactor in which the fuel is deployed.
\end{itemize}

The nuclear fuel cycle begins with uranium mining and ends with the disposal of nuclear waste \cite{fuel_cycle_book}, although some steps may not apply to every fuel cycle.

\section{Frontend of Fuel Cycle}

The steps involving mining, milling, conversion, enrichment, and fuel fabrication correspond to the ``frontend'' of the nuclear fuel cycle. Uranium-rich minerals are radioactive primarily due to the daughter products derived from radioactive decays of uranium. In 2019, global uranium production was approximately 54,750 tons, with most of the mined uranium being used as fuel for nuclear power plants \cite{fuel_cycle_book}.

Uranium recovery is achieved through extraction from ores, followed by concentration and purification. This process involves both excavation and in-situ leaching (ISL). Typically, open-pit mining is used for deposits near the surface, while underground mining is applied for deeper deposits \cite{fuel_cycle_book}. The mined uranium ore is then processed by grinding, followed by uranium leaching using either alkaline or acidic methods. The milling process yields ``yellowcake'', which contains uranium in the form of \(U_3O_8\), with a uranium content greater than \(80\%\) \cite{fuel_cycle_book}.

Approximately 200 tons of \(U_3O_8\) are required to fuel a 1000 MWe nuclear power reactor for one year. The \(U_3O_8\) produced from the uranium mill is further enriched to increase the \(U^{235}\) content from \(0.72\%\) to between \(3\%\) and \(5\%\), which is required for light water reactors. The enrichment process involves converting uranium into uranium hexafluoride (\(UF_6\)), which is solid at room temperature but sublimates at 56.5°C, allowing it to be used in isotope separation.

The primary method used for isotope enrichment today is centrifugation, where thousands of rapidly spinning vertical tubes exploit the small mass difference between the uranium isotopes' hexafluorides, leading to their separation \cite{fuel_cycle_book}.

However the \(UF_6\) is not suitable to be used as fuel in a nuclear reactors, hence needs to be converted to ceramic pellets of \(UO_2\) sintered at temperatures over the \(1400^{\circ}C\). Other pallets composed by a mixed of U, Pu oxide (MOX) are fabricated \cite{fuel_cycle_book}. Currently, conventional UOX fuel have an additive less than \(10\%\) of thorium . This increases the thermal distribution by reducing the need of poison in the reactor \cite{Th_cycle_viability}.

\section{Backend of Fuel Cycle}

The processes that occur after the discharge of irradiated fuel from the reactor, such as the temporary storage of spent fuel, reprocessing, and waste management, are collectively known as the ``backend'' of the nuclear fuel cycle. The energy realised by fission extracted from the fuel is measured as the ``burn up'' \cite{fuel_cycle_book}. The most commonly used metric for fuel burn-up is the amount of fission energy produced per unit mass of fuel. This is expressed as the total energy released, measured in megawatt-days, divided by the initial mass of fuel, including both fissile and fertile materials, and is referred to as \textit{megawatt-days per ton} \((MWd/T)\) \cite{nuclear_reactors_adv}. Since natural uranium contains only\( 0.72\%\) of the fissile isotope \(\prescript{235}{}{U}\), the burn-up of fuel based on natural uranium is expected to be below \(0.72 atom\%\). However, due to the breading of \(\prescript{239}{}{Pu}\) from neutron absorption in \(\prescript{238}{}{U}\), burn-up can achieve values of up to \(1 \text{atom} \%\) \((10,000 MWd/T)\) \cite{fuel_cycle_book}. In typical pressurized heavy water reactors (PHWRs), the burn-up level is around \(7000 MWd/T\). For light water reactors (LWRs) using enriched uranium, burn-up levels can reach \(6–7 \text{atom} \%\), while in fast reactors, burn-up can exceed \(10 \text{atom} \%\). In some cases, burn-up as high as \(20 \text{atom} \%\) has been achieved \cite{fuel_cycle_book}.



\section{Types of cycles}

\section{Closed cycles}
