\chapter{Nuclear Fuel Cycles}

Now we will examine the life cycle of the materials used as fuel in nuclear reactors. The fuel cycle spans from the mining of the ore to the fabrication of the fuel, its irradiation in the reactor, and the subsequent processing of the spent nuclear fuel \cite{fuel_cycle_book}. The nuclear fuel cycle can vary depending on several factors, such as:

\begin{itemize}
    \item The fissile nuclei being used for energy generation, such as any of the isotopes mentioned in \textbf{Section} \ref{sec:possible_fuels}.
    \item The form in which the fuel is used.
    \item The type of reactor in which the fuel is deployed.
\end{itemize}

The nuclear fuel cycle begins with uranium mining and ends with the disposal of nuclear waste \cite{fuel_cycle_book}, although some steps may not apply to every fuel cycle.

\section{Frontend of Fuel Cycle}

The steps involving mining, milling, conversion, enrichment, and fuel fabrication correspond to the ``frontend'' of the nuclear fuel cycle. Uranium-rich minerals are radioactive primarily due to the daughter products derived from radioactive decays of uranium. In 2019, global uranium production was approximately 54,750 tons, with most of the mined uranium being used as fuel for nuclear power plants \cite{fuel_cycle_book}.

Uranium recovery is achieved through extraction from ores, followed by concentration and purification. This process involves both excavation and in-situ leaching (ISL). Typically, open-pit mining is used for deposits near the surface, while underground mining is applied for deeper deposits \cite{fuel_cycle_book}. The mined uranium ore is then processed by grinding, followed by uranium leaching using either alkaline or acidic methods. The milling process yields ``yellowcake'', which contains uranium in the form of \(U_3O_8\), with a uranium content greater than \(80 \, \%\) \cite{fuel_cycle_book}.

Approximately 200 tons of \(U_3O_8\) are required to fuel a 1000 MWe nuclear power reactor for one year. The \(U_3O_8\) produced from the uranium mill is further enriched to increase the \(U^{235}\) content from \(0.72 \, \%\) to between \(3 \, \%\) and \(5 \, \%\), which is required for light water reactors. The enrichment process involves converting uranium into uranium hexafluoride (\(UF_6\)), which is solid at room temperature but sublimates at 56.5°C, allowing it to be used in isotope separation \cite{fuel_cycle_book}.

The primary method used for isotope enrichment today is centrifugation, where thousands of rapidly spinning vertical tubes exploit the small mass difference between the uranium isotope hexafluorides, leading to their separation \cite{fuel_cycle_book}.

However, the \(UF_6\) is not suitable to be used as fuel in a nuclear reactor, hence it must be converted into ceramic pellets of \(UO_2\) sintered at temperatures over the \(1400^{\circ}C\). Other pellets composed by a mixed of U, Pu oxide (MOX) are also fabricated \cite{fuel_cycle_book}. Currently, conventional Uranium Oxide (UOX) fuel have an additive less than \(10 \, \%\) of thorium. This increases the thermal distribution by reducing the need of burnable poison isotopes in the reactor \cite{Th_cycle_viability}.

\section{Backend of Fuel Cycle}

The processes that occur after the discharge of irradiated fuel from the reactor, such as the temporary storage of spent fuel, reprocessing, and waste management, are collectively known as the ``backend'' of the nuclear fuel cycle. The energy released by fission extracted from the fuel is known as the ``burn-up'' \cite{fuel_cycle_book}.

The burn-up of nuclear fuel is defined as the energy produced per unit mass of fuel and is related to the inventory of fission products present in the fuel. It affects both physical and material properties and is an important parameter for the design and operation of power reactors from a safety point of view. Different units of burn-up include megawatt-days per ton of heavy metal (MWd/THM), atom percent (atom \%), and fissions per cubic centimeter (fissions/cc) \cite{fuel_cycle_book}.

The calculation of burn-up is commonly defined by comparing the number of fissioned atoms to the number of heavy metal (HM) atoms present before irradiation in a given fuel. This widely adopted definition provides a clear metric for assessing the extent of fuel utilization \cite{fuel_cycle_book}:

\begin{flalign*}
    &&BU(\text{atom}\%) = \frac{\text{number of fissioned atoms}}{\text{pre-irradiation number of HM atoms}} \times 100 &&
\end{flalign*}

The most commonly used metric for fuel burn-up is the amount of fission energy produced per unit mass of fuel. This is expressed as the total energy released, measured in megawatt-days, divided by the initial mass of fuel, including both fissile and fertile materials, and is referred to as \textit{megawatt-days per ton} \((MWd/T)\) \cite{nuclear_reactors_adv}. Since natural uranium contains only \(0.72 \, \%\) of the fissile isotope \(\prescript{235}{}{U}\), the burn-up of fuel based on natural uranium is expected to be below \(0.72 \, atom \, \%\). However, due to the breeding of \(\prescript{239}{}{Pu}\) from neutron absorption in \(\prescript{238}{}{U}\), burn-up can achieve values of up to \(1 \, \text{atom} \%\) \((10,000 \, MWd/T)\) \cite{fuel_cycle_book}.

In typical pressurized heavy water reactors (PHWRs), the burn-up level is around \(7000 \, MWd/T\). For light water reactors (LWRs) using enriched uranium, burn-up levels can reach \(6–7 \, \text{atom} \, \%\), while in fast reactors, burn-up can exceed \(10 \, \text{atom} \%\), with some cases achieving burn-up as high as \(20 \, \text{atom} \%\) \cite{fuel_cycle_book}.

However, fuel burn-up is limited by the accumulation of fission products and actinides, which absorb neutrons and negatively impact neutron economy, leading to a decrease in the reactor's efficiency. This necessitates the removal of the fuel from the reactor. Nevertheless, a typical fuel based on low enrichment uranium (LEU) still contains \(96 \, \%\) of its original uranium content, approximately \(1 \, \%\) of plutonium, and \(3 \, \%\) of fission products and minor actinides \cite{fuel_cycle_book}. Thus, regardless of the type of fuel or reactor, the irradiated fuel remains a valuable source of fissile materials. Even if the uranium is depleted in terms of its fissile content, it can still be used to prepare fuel for a fast reactor by adding plutonium. Therefore, the fuel discharged from a nuclear reactor, although commonly referred to as ``spent fuel'', is not truly ``spent'' as it still contains fissionable materials \cite{fuel_cycle_book}. 

\subsection{Reprocessing}

Methods for reprocessing discharged fuel have been developed and used either at the research stage or commercial scale for uranium and thorium based fuels. Some methods are \cite{Th_cycle_viability}:

\begin{itemize}
    \item Pyrometallurgy: This method uses heat to initiate the separation of metals from their minerals. 
    \item Electrometallurgy: Also known as pyroprocessing; this technique employs electricity to initiate the separation of metals. However, it is still in the research stage \cite{Th_cycle_viability}.
    \item Hydrometallurgy: This method involves the use of an aqueous solution to dissolve the metal content. In the case of uranium and plutonium extraction, hydrometallurgy uses a solution containing \(30\%\) tri-n-butyl phosphate (TBP) in an aliphatic hydrocarbon diluent, such as kerosene or a mixture of normal paraffinic hydrocarbons, to preferentially extract U and Pu. Popularly known as PUREX (Plutonium Uranium Redox EXtraction) \cite{fuel_cycle_book}.
\end{itemize}

\subsection{Waste Management}

Radioactive waste management is a crucial aspect of the nuclear energy program. The common strategy for managing radioactive waste involves vitrification, where solid waste oxides are encapsulated in borosilicate glass blocks. These glass blocks are then buried in deep geological repositories (DGRs). However, this disposal method requires long-term surveillance due to the very long half-lives of minor actinides (such as neptunium, americium, and curium) and some fission products, making the waste management program highly expensive \cite{fuel_cycle_book}.

The decay of buried radionuclides takes millions of years, and it takes a significant amount of time for the radiotoxicity of long-lived radionuclides present in the waste blocks to reduce to a level comparable to that of natural uranium\cite{fuel_cycle_book}. To address these challenges, processes for partitioning minor actinides, which means separating elements such as neptunium (Np), americium (Am), and curium (Cm) from the spent fuel, as well as fission products, have been developed. These processes aim to reduce the average exposure to operating personnel and can be followed by the transmutation of long-lived radionuclides in fast reactors or accelerator-driven sub-critical systems (ADS) \cite{fuel_cycle_book}.

The strategy of ``Actinide Partitioning'' combined with other emerging strategies such as ``lanthanide-actinide separation'' and ``Am-Cm separation'' offers additional benefits, including the recovery of valuable materials like americium (Am) and curium (Cm) \cite{fuel_cycle_book}.

Actinides are a series of heavy elements in the periodic table, ranging from actinium (Ac) to lawrencium (Lr). They include important nuclear materials such as uranium (U) and plutonium (Pu), as well as minor actinides like neptunium (Np), americium (Am), and curium (Cm). Minor actinides are those actinides that are present in spent nuclear fuel (SNF) in smaller quantities compared to uranium and plutonium.

Lanthanides, on the other hand, are a group of 15 metallic elements from lanthanum (La) to lutetium (Lu) in the periodic table. They are chemically similar to actinides and often found together in SNF, making their separation a challenging but necessary process for effective nuclear waste management.

In the long term, the ``Partitioning \& Transmutation'' (\(P\&T\)) strategy effectively addresses concerns about radioactive waste management. This strategy involves separating long-lived radionuclides, such as minor actinides, from the waste and transmuting them into shorter-lived or stable isotopes. By doing so, \(P\&T\) provides a potential solution for reducing the radiotoxicity and heat load of long-lived radionuclides, thereby improving the safety and sustainability of nuclear waste disposal \cite{fuel_cycle_book}.

\section{Fuel Cycles}

The nuclear fuel cycle can be classified into two main categories: open and closed fuel cycles. The open fuel cycle, also known as the "once-through" fuel cycle, is characterized by the use of nuclear fuel without reprocessing. In this cycle, the fuel is used until it is no longer economically viable to extract energy from it, after which it is disposed of as waste \cite{fuel_cycle_book}.

\subsection{Open Fuel Cycle}

The open fuel cycle is the simplest and most common fuel cycle used in commercial nuclear power plants. In this cycle, the fuel is used in the reactor only once, then discharged and stored in interim storage facility where it is kept until most of the short-lived material has decayed. The spent fuel is stored in a DGR for long-term disposal. The open fuel cycle is used in the USA, UK and South Africa \cite{fuel_cycle_book}.

\subsubsection{Twice-Through Fuel Cycle}

This cycle reprocess the spent fuel to recover uranium and plutonium for reuse in the reactor. MOX is fabricated from the reprocessed fuel and then used in the reactor for a second time before being disposed of in a DGR similar to the ``once-through'' cycle \cite{fuel_cycle_book}. The major reason for recycling of the spent nuclear fuel (SNF) in the case of the “twice-through” nuclear fuel cycle (NFC) is due to the fact that the SNF still contains approximately \(96\%\) of the reusable material (U and Pu) and, hence, is still considered as an energy source due to its appreciable fissile material content. Additional cycles in a thermal reactor are not practical due to degradation of the isotopic composition of Pu \cite{fuel_cycle_book}.

As compared to the “once-through” NFC, in which the SNF is stored after use and buried in DGRs following the vitrification process, the “twice-through” NFC can utilize \(17 \, \%\) more natural uranium by the MOX fuel option, which can largely reduce the heavy enrichment costs \cite{fuel_cycle_book}. 

\subsection{Closed Fuel Cycle}

The objective of the closed fuel cycle is a more efficient utilization of fissile materials and the reduction of long-lived radioactive waste. The closed fuel cycle requires the development of advance technologies for reprocessing to achieve a sustainable NFC \cite{fuel_cycle_book}. 

\subsubsection{Closed Fuel Cycle without MA Recovery}

In a closed fuel cycle with minor actinide and fission product recovery, uranium and plutonium are extracted from the irradiated fuel. The separated plutonium can be reused in thermal or fast reactors, while the depleted uranium can be utilized in MOX fuels for both types of reactors. This approach was initially implemented by countries with limited uranium reserves, aiming to recover and utilize the depleted uranium and plutonium from the discharge fuel \cite{fuel_cycle_book}.


\subsubsection{Closed Fuel Cycle with MA Recovery}

The recovery of minor actinides often involves the simultaneous recovery of lanthanides due to their chemical properties' similarity. These recovered minor actinides can then be burned in ADSs or fast reactors. In this context, the separation of lanthanides from actinides, followed by the separation of americium from curium, has become an important step for the effective transmutation of long-lived actinides. It is essential, however, that the recovery of radiotoxic minor actinides be nearly \(100 \, \%\) effective in order to simplify waste immobilization and ensure the remnant waste qualifies as ``non-alpha'' waste for disposal in DGRs \cite{fuel_cycle_book}.

\subsubsection{Closed Fuel Cycle with MA and Fission Product Recovery}

Fission products constitute about \(0.6 \, \%\) of the irradiated fuel, with long-lived fission products making up only around \(0.04 \, \%\). Although they are present in small quantities, the main source of radioactivity in high level liquid waste (HLLW) comes from highly radioactive fission nuclides like \(\prescript{137}{}{Cs}\) and \(\prescript{90}{}{Sr}\), which are responsible for the majority of the radiation dose. 

Other radionuclides, such as \(\prescript{106}{}{Ru}\) with moderate activity, and long-lived fission products like \(\prescript{99}{}{Tc}\), \(\prescript{93}{}{Zr}\), \(\prescript{135}{}{Cs}\), \(\prescript{129}{}{I}\), and \(\prescript{107}{}{Pd}\), also contribute to the overall radioactivity \cite{fuel_cycle_book}.

Some of these fission products, such as \(\prescript{90}{}{Y}\) are useful in nuclear medicine. This has led to efforts to separate these radionuclides from HLLW before vitrification. In addition, long-lived fission products can be transmuted into shorter-lived ones to reduce the time required for monitoring vitrified waste. In India, the current strategy involves separating fission products prior to the actinide partitioning step \cite{fuel_cycle_book}.

Next chapter will delve into the thorium fuel cycle and its potential advantages over the uranium fuel cycle, as well as drawbacks and challenges associated with its implementation. Additionally, it will explore a promising prospect for the fourth generation of nuclear reactors, the molten salt reactor (MSR), which is well-suited for the thorium fuel cycle.