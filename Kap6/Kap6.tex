\chapter{Thorium Fuel Cycle}

Thorium based nuclear fuel have been of interest from 1950 to 1970 \cite{Th_cycle_viability}. This led to the development of multiple experimental reactors that could operate with thorium based fuel \cite{TMSR_book}. In the 1950, thorium was of special interest due to its higher abundance compared to uranium, being three times more abounded than uranium \cite{Th_cycle_viability,TMSR_book}.

The thorium fuel cycle is based on the use of \(\prescript{232}{}{Th}\) as fertile material, which is converted into fissile \(\prescript{233}{}{U}\) by neutron capture and beta decay. This implies that an additional step of converting the fertile isotope into the fissile material is required. The subsequent use of \(\prescript{233}{}{U}\) as fuel is possible in two ways. The first option is an open fuel cycle, based of the breading of \(\prescript{233}{}{U}\) and in situ fission of this isotope, this would not involve chemical separation of \(\prescript{233}{}{U}\) from the irradiated fuel \cite{IAEA_Th_Potential}. The second option is a closed fuel cycle, where the irradiated fuel is reprocessed to separate the fissile material from the rest of the irradiated fuel, this fissile material is then used to fabricate new fuel elements \cite{IAEA_Th_Potential}. 

\section{Considerations for the Thorium Fuel Cycle}

The chemical properties of thorium make thorium dioxide more stable and has higher radiation resistance than uranium dioxide \cite{IAEA_Th_Potential}. \(ThO_2\) has higher thermal conductivity and lower coefficient of thermal expansion compared to \(UO_2\), this might make \(ThO_2\) based fuels to have a better in-pile perform \cite{IAEA_Th_Potential}. Furthermore, \(ThO_2\) is considered inert and does not oxidize unlike \(UO_2\), making long term storage of irradiated fuel easier \cite{IAEA_Th_Potential}.

The thorium fuel cycle has a higher conversion ratio in thermal reactors compared to the uranium fuel cycle, this is due to the fact that \(\prescript{232}{}{Th}\) has a higher neutron capture cross section compared to \(\prescript{238}{}{U}\) with thermal neutrons \cite{IAEA_Th_Potential}. Additionally, the fraction of neutrons liberated per neutron absorbed (\(\eta\)) is greater than \(2.0\) over a wide range of thermal neutron, unlike \(\prescript{235}{}{U}\) and \(\prescript{239}{}{Pu}\). Making \(\prescript{232}{}{Th} - \prescript{233}{}{U}\) operational with fast or thermal spectra. Unlike \(\prescript{238}{}{U} - \prescript{239}{}{Pu}\) which a breading chain reaction can only be obtained with fast neutrons \cite{IAEA_Th_Potential}.

Finally, the thorium fuel cycle has ``intrinsic proliferation resistance'' \cite{IAEA_Th_Potential}. This is because of the formation of \(\prescript{232}{}{U}\) via neutron multiplication reactions (\(n,2n\)) with \(\prescript{232}{}{Th}, \prescript{233}{}{Pa}\) and \(\prescript{233}{}{U}\). \(\prescript{232}{}{U}\) has physical properties, such as a very short half-life (\(t_{1/2} = 68.90 \, \text{years} \)) and the emission of strong gamma radiation by its decay products \cite{IAEA_Th_Potential,NNDC}. This makes the thorium fuel cycle less attractive for the production of nuclear weapons \cite{IAEA_Th_Potential}.

\section{Thorium Front End}

All thorium isotopes have short half-lives, except \(\prescript{232}{}{Th}\) which has a half-life of \(t_{1/2} = 1.405 \times 10^{10} \, \text{years}\) \cite{NNDC}, making all the natural thorium available on earth constituted by \(\prescript{232}{}{Th}\). Generally, Natural thorium is presented in association with other elements, such as rare earths elements and uranium in diverse rocks types such as veins of thorite, thorianite, uranothorite and  monazite in granites and other acidic intrusions \cite{IAEA_Th_Potential}. 

The global thorium reserves to \(2013\) are estimated to be around \(6.24 \, \text{million tons}\). India has the largest thorium reserves, with approximately \(846,000 \, \text{tons}\), followed by Brazil with \(632,000 \, \text{tons}\) and the United States with \(595,000 \, \text{tons}\) \cite{Th_cycle_viability}.

Thorium reserves in Colombia are associated mainly with igneous rocks, such as granites and pegmatites, and can also be found in placers, especially in alluvial deposits \cite{Th_Colombia}. The concentration of thorium in the country varies significantly depending on the geological formation, with values ranging between \(0.1\) and \(585 \, \text{mg}/\text{kg}\) in sediments. The highest concentrations are observed in areas with granitic formations, such as the Sierra Nevada de Santa Marta and parts of the Andean region, which are known for their igneous and metamorphic rocks \cite{Th_Colombia}. 

The mining and extraction of thorium from monazite is relatively straightforward and differs significantly from the extraction of uranium from its ores. Most commercially exploited monazite comes from beach or river sands, often alongside other heavy minerals. Compared to uranium mining, thorium extraction involves much less overburden, and the radioactive waste produced is approximately two orders of magnitude lower. The so-called radon impact is also considerably smaller for thorium mining, due to the shorter half-life of thoron (\(\prescript{220}{}{Rn}\)) compared to radon, leading to simpler tailings management \cite{IAEA_Th_Potential,Thoron}.

The monazite is finely ground and dissolved in concentrated sodium hydroxide at around \(140^{\circ}C\), followed by a series of chemical processes, including solvent extraction and ion exchange, to yield pure thorium nitrate. This is then precipitated as thorium oxalate and calcined to obtain \(ThO_2\) powder. Thorium recovery projects typically separate thorium in pure oxalate form, which is easier to handle and store for future applications, such as preparing mantle-grade thorium nitrate or nuclear-grade thorium oxide. Uranium present in monazite is also separated in the form of crude uranium concentrate during the process \cite{IAEA_Th_Potential}.

\section{Thorium Open Fuel Cycle}

An open fuel cycle avoids the complex engineering requirements associated with reprocessing irradiated fuel and fabricating highly radiotoxic \(\prescript{233}{}{U}\)-based fuels. The core layout is designed such that each fuel assembly comprises a ``seed'' material, typically medium-enriched uranium or plutonium, and ``blanket'' material of thorium \cite{IAEA_Th_Potential}. Separation of seed and blanket, optimization moderator to fuel ratio and long periods of fuel usage offer the possibility that around \(40 \, \%\) of the power generated by fission comes from \(\prescript{233}{}{U}\) \cite{IAEA_Th_Potential}.

This scheme is very attractive for an introduction if thorium in nuclear power reactors because of the direct utilization of \(\prescript{233}{}{U}\), avoiding the need of reprocessing and fabrication of new fuel elements with \(\prescript{233}{}{U}\) \cite{IAEA_Th_Potential}. Furthermore, the once-through thorium fuel cycle opens the possibility of incineration if weapons-grade plutonium in light water reactors \cite{IAEA_Th_Potential}. To achieve this, thorium oxide containing \(5 \, \%\) of \(PuO_2\), could be used as driver fuel. The exclusion of uranium from the fuel composition results in an increase in the rate of plutonium incineration compared to standard MOX fuel \cite{IAEA_Th_Potential}.

Similarly, civil plutonium could be burned in a thorium fuel cycle by using the same combination of thorium oxide and plutonium oxide \cite{IAEA_Th_Potential}. Since, \(\prescript{240}{}{Pu}\) is present is significant quantities in civil grade plutonium, and it is a good burnable absorbed, then there is no need for use burnable absorber in the form of gadolinium, integrated into the fuel \cite{IAEA_Th_Potential}. Plutonium oxide fuel without major modification in the reactor core could be a direct replacement for LEU oxide fuel \cite{IAEA_Th_Potential}.

\section{Thorium Closed Fuel Cycle}

The closed fuel cycle for thorium-based fuels includes essential steps such as reprocessing irradiated fuel and separating converted \(\prescript{233}{}{U}\). In some reactor designs, like LWRs, mixed thorium-plutonium oxide fuel is used to facilitate the conversion of \(\prescript{232}{}{Th}\) to \(\prescript{233}{}{U}\). However, an important consideration in recycling \(\prescript{233}{}{U}\) is the presence of \(\prescript{232}{}{U}\), which introduces radiological challenges due to its strong gamma emissions \cite{IAEA_Th_Potential}.

Various countries have implemented closed thorium fuel cycles. In Russia, fast breeder reactors like the BN-800 have been studied for their capability to achieve self-sufficiency in the \(\prescript{232}{}{Th}-\prescript{233}{}{U}\) cycle with a breeding ratio close to or exceeding 1.0. France has also explored similar concepts in reactor types such as high-temperature gas-cooled reactors (HTGRs) and HWRs, finding that these reactors can approach a breeding ratio of 1.0, though not exceed it \cite{IAEA_Th_Potential}.

India has adopted a strategic three-stage nuclear program to maximize its vast thorium reserves. The first stage, utilizing PHWRs, employs \(\text{ThO}_2\) assemblies to flatten neutron flux. The second stage, using liquid metal-cooled fast breeder reactors (LMFBRs), incorporates \(\text{ThO}_2\) blankets to produce \(\prescript{233}{}{U}\). Finally, the third stage involves advanced thermal reactors that are self-sustaining in \(\prescript{232}{}{Th}-\prescript{233}{}{U}\) fuels, like the Advanced Heavy Water Reactor (AHWR) designed by BARC. This multistage program enables India to establish a sustainable thorium cycle by gradually transitioning to a thorium-based closed fuel cycle \cite{IAEA_Th_Potential}.

\section{Thorium Molten Salt Reactors (TMSR)}